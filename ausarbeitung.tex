% -*- LuaLaTeX-command: "lualatex -shell-escape" -*-

\documentclass[a4paper, fontsize=12pt, toc=bibliographynumbered]{scrreprt}
\setcounter{tocdepth}{3}
\setcounter{secnumdepth}{3}
\usepackage{microtype}
\usepackage[onehalfspacing]{setspace}
\usepackage[defaultlines=2, all]{nowidow}
\usepackage[left=4cm, right=2cm, top=2.5cm, bottom=2.5cm]{geometry}
\usepackage{minted}
\usepackage{tabu}
\usepackage{enumitem}
\setlist{nolistsep}
\usepackage[footnote]{acronym}
\usepackage[babelshorthands]{polyglossia}
\setmainlanguage{german}
\usepackage[colorlinks=true, linktoc=page, urlcolor=blue, citecolor=Green]{hyperref}

\newcommand{\person}[2]{#1 \textsc{#2}}
\newcommand{\mono}[1]{\texttt{#1}}

\newcommand{\abbildung}[3]{\begin{figure}[H]\centering
    \fbox{\includegraphics[width=#1\textwidth]{#2}}\caption{#3}
    \label{fig:#2}\end{figure}}
\begin{document}

% Akronyme

\newacro{OSC}{Open Sound Control}
\newacro{MIDI}{Musical Instrument Digital Interface}
\newacro{GPIO}{General Purpose Input Output}
\newacro{SDK}{Software Development Kit}
\newacro{API}{Application Programming Interface}
\newacro{LAN}{Local Area Network}
\newacro{DHCP}{Dynamic Host Configuration Protocol}
\newacro{DNS}{Domain Name System}
\newacro{HTTP}{Hyper-Text Transfer Protocol}
\newacro{JSON}{JavaScript Object Notation}
\newacro{SSH}{Secure SHell}
\newacro{LFO}{Low-Frequency Oscillator}

\setmainfont{Tex Gyre Termes}
\setsansfont{Tex Gyre Heros}
\setmonofont{Tex Gyre Cursor}[Ligatures={NoRequired,NoCommon,NoContextual}]
\setkomafont{chapterentry}{\bfseries}

\renewcommand\listoflistingscaption{Codeverzeichnis}

\begin{titlepage}
  \begin{center}
    \textsc{\large Fachhochschule der Wirtschaft\\FHDW}\\[1em]
    \textsc{\large Bergisch Gladbach}\\[2em]
    \textsc{Studienarbeit}\\[6em]
    {\LARGE Digitales Theremin mit Raspberry Pi und Leap Motion}\\[20em]
    \begin{tabu} to 0.8\textwidth {X[l] X[r]}
      \emph{Autor:}\linebreak
      \person{Vasilij}{Schneidermann}
      &
      \emph{Prüfer:}\linebreak
      \person{Dr.~Peter}{Tutt}
    \end{tabu}
    \vfill
    \emph{Studiengang:}\\
    Wirtschaftsinformatik\\[1em]
    \emph{Fachbereich:}\\
    Consulting\\[2em]
    \emph{Abgabetermin:}\\
    07.~April 2016
  \end{center}
\end{titlepage}

\pagenumbering{roman}

\tableofcontents
\listoflistings
\clearpage
\pagenumbering{arabic}
\setcounter{page}{1}

\chapter{Einführung}

Ähnlich wie der populäre Arduino-Microcontroller hat der Raspberry Pi
Kleinstcomputer der breiten Öffentlichkeit näher gebracht.  Mögliche
Einsatzgebiete sind stromsparende Medienserver, Verwendung im
Unterricht und Experimente mit dem \emph{Internet of Things}.  Die
vorliegende Studienarbeit konzentriert sich auf ein etwas
ungewöhnlicheres Projekt in dessen Rahmen ein digitales Theremin unter
Zuhilfenahme einer Leap Motion als kontaktloses Eingabegerät umgesetzt
wurde.

\chapter{Grundlagen}

\section{Raspberry Pi}

Der Raspberry Pi ist ein im Jahr 2011 eingeführter
Einplatinencomputer\footnote{BBC (2011), Online im Internet}.  Ziel
seiner Einführung war die Verwendung im Unterricht und
Entwicklungsländern.

Das verwendete Gerät weist unter anderem USB-Ports, \ac{GPIO}-Pins,
einen Ethernet-Port, HDMI-Anschluss und einen
SD-Kartenleser\footnote{Raspberry Pi Foundation (2016), Online im
  Internet}. Letzterer wird verwendet um ein Betriebssystem, in
unserem Falle Raspbian\footnote{Raspbian (2016), Online im Internet}
zu starten.

\section{Leap Motion}

Die Leap Motion ist ein Eingabegerät welches es mithilfe von Kameras
und Infrarotsensoren ermöglicht hochauflösend Handbewegungen zu
verfolgen\footnote{Leap Motion Inc. (2016), Online im Internet}.  Die
dabei anfallenden Daten können nach Installation des offiziellen
\ac{SDK} mithilfe einer \ac{API} für eine Programmiersprache
ausgewertet werden.

\section{Python}

Python ist eine populäre dynamische Programmiersprache\footnote{Python
  Software Foundation (2016), Online im Internet}.  Eines der Ziele
von Python ist es Programme lesbar und leicht verständlich zu
gestalten. Die Sprache unterstützt verschiedenste
Programmierparadigmen, hat ein großes Ökosystem mit über $75,000$
Modulen\footnote{PyPi (2016), Online im Internet} und ist
insbesondere im wissenschaftlichen Umfeld beliebt.

\section{ChucK}

ChucK ist eine Programmiersprache für die Synthese, Komposition und
Analyse von Musik\footnote{ChucK (2016), Online im Internet}.  Eine
Besonderheit im Vergleich zu anderen Programmiersprachen in diesem
Bereich ist der starke Fokus auf Zeit: Um Ton ausgeben zu können muss
der Programmierer explizit Zeit verstreichen lassen.  Diese
ungewöhnliche Designentscheidung ermöglicht es individuelle
Komponenten der Synthese frei miteinander zu synchronisieren und auf
das der Situation angemessene Detaillevel herunterzugehen.

\section{\ac{OSC}}

OSC ist ein im Audiobereich breit verwendetes Protokoll und Format zur
Kommunikation zwischen Instrumenten\footnote{Open Sound Control
  (2016), Online im Internet}.  Im Gegensatz zu dem populären
\ac{MIDI}-Standard ist es deutlich leichtgewichtiger, allgemeiner und
einfacher zu implementieren.

\section{Theremin}

Das Theremin ist ein elektronisches Musikinstrument welches mit den
Händen berührungsfrei gespielt wird.  Es ist nach seinem Erfinder,
\person{Léon}{Theremin} benannt.  Die linke Hand kontrolliert die
Lautstärke, die rechte Hand die Tonhöhe.

\chapter{Planung und Übersicht}

Die anfängliche Idee die Leap Motion direkt an den Raspberry Pi
anzuschließen wurde verworfen, da laut offizieller Stellungsnahme des
Herstellers der Raspberry Pi nicht die nötigen Mindestanforderungen
erfüllt\footnote{Leap Motion Community (2013), Online im Internet}.
Stattdessen wird in einem Blogpost für ein vergleichbares
Projekt\footnote{Leap Motion Blog (2015), Online im Internet}
vorgeschlagen die Leap Motion an einem leistungstärkerem Rechner zu
betreiben und den Raspberry Pi die benötigten Daten periodisch
abfragen zu lassen.  Für die Kommunikation müssten dann ein Client und
ein Server geschrieben werden welche ein gemeinsames Protokoll
sprechen.  Zur Verbindung von dem Rechner und Raspberry Pi wird ein
Netzwerk mithilfe eines Switch und zwei \ac{LAN}-Kabeln aufgebaut.

Da die offizielle Anbindung an die GPIOs\footnote{PyPi RPi.GPIO
  (2016), Online im Internet} in Python geschrieben ist und die Leap
Motion auch eine Python-\ac{API}\footnote{Leap Motion Developer
  (2016), Online im Internet} aufweist, wurde der Einfachheit halber
Python für beide Komponenten verwendet.  Als Protokoll wird \ac{HTTP}
mit \ac{JSON} als \emph{Payload} verwendet, für die Server-Seite ist
Flask\footnote{Armin Ronacher (2016), Online im Internet} und für die
Client-Seite Requests\footnote{Kenneth Reitz (2016), Online im
  Internet} zuständig.

Das andere Problem ist die Audiosynthese.  Zwar ist es möglich eine
vorhandene Audiobibliothek zu nutzen und den Ton Sample für Sample zu
konstruieren, jedoch wurde diese Vorgehensweise als zu aufwändig für
das Ziel eingestuft\footnote{Wäre das Ziel hingegen
  DSP-Programmierung, würde dieser Ansatz viel mehr Sinn ergeben.}.
Aufgrund dessen wurden stattdessen drei verschiedene
Programmiersprachen für diesen Zweck entwickelte evaluiert,
SuperCollider\footnote{SuperCollider (2016), Online im Internet},
Csound\footnote{Csound (2016), Online im Internet} und
ChucK\footnote{ChucK (2016), Online im Internet}.  Super Collider war
die anfangs favorisierte Lösung, wurde aber wegen mangelnder
Dokumentation für konsolenbasierte Steuerung und einer
Inkompatibilität auf der ARM-Architektur des Raspberry Pi schließlich
verworfen.  Csound sieht sehr spannend aus, kam aber wegen des Fokus
auf orchestralen Kompositionen nicht in Frage.  ChucK hat brauchbare
Dokumentation und funktionierte mit wenig Setup auf dem Raspberry Pi
und wurde deswegen ausgewählt.

Für die Modulation von Tonhöhe und Lautstärke ist ebenfalls ein
Protokoll nötig.  ChucK unterstützt das Empfangen von
\ac{OSC}-Nachrichten nativ, deswegen wird nur noch
pyOSC\footnote{pyOSC (2010), Online im Internet} auf der Client-Seite
benötigt.

Der Aufbau lässt sich in folgender Grafik zusammenfassen:

\abbildung{0.8}{digitales-theremin}{Digitales Theremin, Quelle: Eigene Abbildung}

\chapter{Umsetzung}

\section{Netzwerksetup}

Der Raspberry Pi wird ohne vorinstalliertes Betriebssystem geliefert,
daher ist es notwendig zuerst ein \emph{Image} herunterzuladen und es
auf eine SD-Karte zu schreiben.  Die Quellen und Anleitungen dafür
werden vom Raspbian-Projekt
bereitgestellt\footnote{\url{https://www.raspbian.org/RaspbianImages}},
einem Debian\footnote{\url{https://www.debian.org/}}-Derivat für den
Raspberry Pi.  Um den Computer bedienen zu können, ist es notwendig
Tastatur und Monitor anzuschließen.

Das Netzwerk ist sehr einfach aufgebaut.  Dem Raspberry Pi wird die
IP-Adresse \mono{192.168.178.23} und dem Laptop \mono{192.168.178.24}
zugewiesen\footnote{Wichtig hier ist darauf zu achten, dass diese
  Adressen noch nicht im Netzwerk vergeben sind oder noch besser, Teil
  eines anderen als des aktuell verwendeten lokalen Netzwerks sind.}.
Unter Zuhilfenahme eines \ac{DHCP}-Servers und der von Google
bereitgestellten Adresse \mono{8.8.8.8} für die \ac{DNS}-Auflösung
kann die Konfiguration auf ein Minimum beschränkt werden.  Fügt man
Port Forwarding und Masquerading hinzu, kann der Raspberry Pi die
Internetverbindung vom Laptop mitverwenden was z.B.~für die
Installation neuer Pakete äußerst hilfreich ist.

Die Konfiguration wird mithilfe des \mono{ping}-Tools
getestet: Es sollte möglich sein von beiden Seiten aus die andere
Seite zu erreichen.  Im Anschluss dessen kann man sich vom Laptop aus
mit \mono{ssh pi@192.168.178.23} auf den Raspberry Pi über \ac{SSH}
verbinden\footnote{Annahme hier ist dass der Benutzername auch
  \mono{pi} ist}.

Schnitzer bei der Konfiguration können ausgebessert werden indem man
die SD-Karte herausnimmt, in einen anderen Rechner einbindet und die
Datei \mono{/boot/cmdline.txt} editiert.  Um z.B.~mit einer festen
IP-Adresse erreichbar zu sein, genügt es den Parameter
\mono{ip=<address>} hinzuzufügen.  Dieser kann anschließend nach
erfolgreicher Reparatur des Netzwerksetups für einen normalen
Bootvorgang wieder entfernt werden.

\section{Handtracking}

Die Leap Motion-API bietet zwei Modi an: Event-basiert (für die
Verarbeitung aller Daten) und Polling (bedarfsgesteuerte Verarbeitung
von Daten).  Da nur ein Bruchteil der Daten periodisch angefragt wird,
ist letzterer Ansatz sinnvoller.  Folgender Code ist in Aufarbeitung
und Bereitstellung der Daten aufgeteilt:

\begin{minted}{python}
import leap
import logging
from flask import Flask, jsonify

log = logging.getLogger('werkzeug')
log.setLevel(logging.ERROR)

app = Flask(__name__)
controller = leap.Controller()

def hand_state():
    frame = controller.frame()
    hands = frame.hands
    state = {'left': None, 'right': None}

    for hand in hands:
        if hand.is_valid:
            if hand.is_left:
                state['left'] = hand.palm_position.y
            elif hand.is_right:
                state['right'] = hand.palm_position.y

    return state

@app.route('/')
def api_root():
    return jsonify(hand_state())

def main():
    app.run(host='0.0.0.0')

if __name__ == '__main__':
    main()
\end{minted}

Anschließend kann der Server mithilfe von \mono{python laptop.py}
gestartet werden.  In einem anderen Terminal kann man die
Funktionalität dessen durch \mono{curl http://localhost:5000} testen.
Je nach Handstatus sollte JSON mit relativen Distanzen oder dem
Sonderwert \mono{null} für eine nicht vorhandene Hand zurückgegeben
werden.

\section{Visuelles Feedback}

Um mehr Rückmeldung vom Raspberry Pi erhalten zu können wird ein
\emph{Breadboard} mit zwei Leuchtdioden-Schaltkreisen bestückt:

\abbildung{0.8}{LED}{LED-Schaltkreis, Quelle: Eigene Abbildung}

Die Idee dabei ist, dass man beim Spielen des Instruments informiert
wird ob die Hand zur Steuerung der Tonhöhe sich im validen Bereich
befindet oder darüber hinaus geht.  Im ersteren Fall wird eine grüne
Leuchtdiode angeschaltet und eine rote Leuchtdiode ausgeschaltet,
ansonst gilt das gleiche umgekehrt.  Der Raspberry Pi ermöglicht es
mit der Mehrzahl seiner GPIO-Anschlüsse einen elektrischen Schalter zu
implementieren.  Folgender Code geht davon aus, dass einmal Pin 5 und
6 (BCM 3) sowie Pin 40 und 39 (BCM 21) angeschlossen sind\footnote{Zur
  Lokalisierung der Pins ist \url{http://pinout.xyz/} hilfreich}:

\begin{minted}{python}
import RPi.GPIO as GPIO
import requests
import time

from OSC import OSCMessage
from OSC import OSCClient

RED_LED = 3
GREEN_LED = 21
SERVER = 'http://192.168.178.24:5000'

def led_setup():
    GPIO.setmode(GPIO.BCM)
    GPIO.setwarnings(False)
    GPIO.setup(RED_LED, GPIO.OUT)
    GPIO.setup(GREEN_LED, GPIO.OUT)

def red_led_on():
    GPIO.output(RED_LED, 1)

def red_led_off():
    GPIO.output(RED_LED, 0)

def green_led_on():
    GPIO.output(GREEN_LED, 1)

def green_led_off():
    GPIO.output(GREEN_LED, 0)

def led_teardown():
    GPIO.cleanup()

def hand_state():
    return requests.get(SERVER).json()

def led_control(state):
    distance = state['right']
    if distance and distance > 100 and distance < 600:
        red_led_off()
        green_led_on()
    else:
        red_led_on()
        green_led_off()

def clamp(n, lower, upper):
    if n < lower:
        return 0.0
    elif n >= lower and n < upper:
        return (n - lower) / float(upper)
    else:
        return 1.0

def main():
    led_setup()
    client = OSCClient()
    address = ('127.0.0.1', 6666)
    while True:
        state = hand_state()
        led_control(state)
        if state['left']:
            msg = OSCMessage('/volume')
            msg.append(clamp(state['left'], 100, 600))
            client.sendto(msg, address)
        if state['right]:
            msg = OSCMessage('/pitch)
            msg.append(clamp(state['right], 100, 600))
            client.sendto(msg, address)
        time.sleep(0.01)
    led_teardown()

if __name__ == '__main__':
    main()
\end{minted}

Ähnlich wie das vorige Code-Beispiel wird dieses mit \mono{python
  raspberry.py} ausgeführt.  Funktioniert alles, kann man die grüne
Leuchtdiode durch das Bewegen der rechten Hand über den Sensor zum
Leuchten bringen.

Das PyOSC-Modul ist nicht direkt aus den Raspbian-Paketarchiven
installierbar, deswegen ist eine manuelle Installation notwendig:

\begin{minted}{shell-session}
$ curl -Ok https://trac.v2.nl/raw-attachment/wiki/pyOSC/\
pyOSC-0.3.5b-5294.tar.gz
$ cd pyOSC-0.3.5b-5294
$ sudo ./setup.py install
\end{minted}

\section{Audiosynthese}

Es ist anzumerken, dass ChucK unter Raspbian nicht aktuell genug für
das nachfolgende Beispiel ist.  Für die manuelle Installation sind
folgende Schritte nötig:

\begin{minted}{shell-session}
$ curl -O \
  http://chuck.cs.princeton.edu/release/files/chuck-1.3.5.2.tgz
$ tar -xf chuck-1.3.5.2
$ cd chuck-1.3.5.2/src
$ sudo apt-get install gcc bison flex libasound2 \
  libasound2-dev libsndfile1 libsndfile1-dev
$ make linux-alsa
$ sudo make install
\end{minted}

Bei genauerer Beobachtung eines Theremins fällt auf, dass es mehr
Anforderungen als die Steuerbarkeit von Lautstärke und Tonhöhe mit der
Hand gibt:

\begin{itemize}
\item Reine, stumpfe Klangfarbe
\item Niederfrequente Modulation der Tonhöhe
\item Linearisiertes Frequenzfeld
\item Übergangslose Änderung der Tonhöhe
\end{itemize}

Eine Umsetzung aller Punkte findet sich in folgendem Code-Beispiel in
ChucK:

\begin{minted}{c}
SinOsc sin => dac;
0.5 => dac.gain;
440.0 => float oldFreq;
440.0 => float newFreq;
440.0 => sin.freq;

SinOsc lfo => blackhole;
5 => lfo.freq;

OscIn oscpitch;
6666 => oscpitch.port;
"/pitch" => oscpitch.addAddress;

OscIn oscvol;
6666 => oscvol.port;
"/volume" => oscvol.addAddress;

fun float fmin(float a, float b) {
    return (a < b) ? a : b;
}

fun void interpolate() {
    1.5 => float step;
    while (true) {
        fmin(oldFreq, newFreq) + Math.fabs(oldFreq - newFreq) /
            step => oldFreq;

        oldFreq => sin.freq;
        5::ms => now;
    }
}

fun void wobble() {
    while (true) {
        lfo.last() * 0.01 * oldFreq +=> oldFreq;
        10::ms => now;
    }
}

fun void adjustVolume() {
    while (true) {
        oscvol => now;
        while (oscvol.recv(OscMsg msg)) {
            Std.fabs(msg.getFloat(0)) => dac.gain;
        }
    }
}

spork ~ interpolate();
spork ~ wobble();
spork ~ adjustVolume();

while (true) {
    oscpitch => now;
    while (oscpitch.recv(OscMsg msg)) {
        Std.fabs(msg.getFloat(0)) => float pitch;
        Math.pow(2, pitch * 2 + 8) => newFreq;
    }
}
\end{minted}

Zum Testen führt man das obige Code-Beispiel mit \mono{chuck
  theremin.ck} aus.  Sollte nichts zu hören sein, kann es sein, dass
ChucK nicht die richtige Soundkarte erkannt hat.  Man kann alle
bekannten mithilfe von \mono{chuck --probe} auflisten, ist
z.B.~Soundkarte 3 die erwünschte, verwendet man \mono{chuck --dac3
  theremin.ck}.

Die Klangfarbe wurde mithilfe eines Sinusoszillators angenähert.  Dies
entspricht nicht ganz einem authentischen Theremin\footnote{Dafür wäre
  eine leichte Verzerrung der Wellenform notwendig, diese wurde der
  Einfachheit halber ausgelassen}, nähert dieses aber hinreichend an.
Für die niederfrequente Modulation wird ein \ac{LFO} verwendet
welcher die Tonhöhe mit einer Frequenz von 5~hz moduliert.

Die Linearisierung des Frequenzfeldes klingt schwierig, ist aber
leicht umsetzbar wenn man bedenkt, dass eine Verdopplung der Frequenz
einen Oktavensprung bedeutet.  Im obigen Beispiel werden die
Eingabewerte zwischen $0.0$ und $1.0$ auf Frequenzen zwischen $2^8 hz$
und $2^{10} hz$ verteilt.

Die Übergänge werden mithilfe einer einfachen Interpolation zwischen
den Zeitpunkten jedes eingehenden OSC-Pakets geglättet.  Man startet
mit einer Ausgangsfrequenz, das OSC-Paket setzt eine Zielfrequenz und
die Interpolation setzt die Ausgangsfrequenz auf einen Zwischenwert.

\chapter{Ehrenwörtliche Erklärung}

Hiermit erkläre ich, dass ich die vorliegende Studienarbeit
selbständig angefertigt habe.  Es wurden nur die in der Arbeit
ausdrücklich benannten Quellen und Hilfsmittel benutzt.  Wörtlich oder
sinngemäß übernommenes Gedankengut habe ich als solches kenntlich
gemacht. Diese Arbeit hat in gleicher oder ähnlicher Form noch keiner
Prüfungsbehörde vorgelegen.

Vasilij Schneidermann

Bergisch Gladbach, den 7.~April 2016.

% \chapter{Quellenverzeichnis}
\renewcommand{\bibname}{Quellenverzeichnis}
\begin{thebibliography}{9}

  \BreakBibliography{\minisec{Internetquellen}}
\bibitem{bbc}BBC (2011), \emph{BBC - dot.Rory: A 15 pound computer to
    inspire young programmers}, Online im Internet:
  \url{http://www.bbc.co.uk/blogs/thereporters/rorycellanjones/2011/05/a_15_computer_to_inspire_young.html},
  Stand: 07.04.2016
\bibitem{rpi2-specs}Raspberry Pi Foundation (2016), \emph{Raspberry Pi
    2 Model B}, Online im Internet:
  \url{https://www.raspberrypi.org/products/raspberry-pi-2-model-b/},
  Stand: 07.04.2016
\bibitem{raspbian}Raspbian (2016), \emph{FrontPage - Raspbian}, Online
  im Internet: \url{https://www.raspbian.org/}, Stand: 07.04.2016
\bibitem{leapmotion}Leap Motion Inc. (2016), \emph{Leap Motion | Mac
    \& PC Motion Controller for Games, Design, Virtual Reality \&
    More}, Online im Internet: \url{https://www.leapmotion.com/},
  Stand: 07.04.2016
\bibitem{python}Python Software Foundation (2016), \emph{Welcome to
    Python.org}, Online im Internet: \url{https://www.python.org/},
  Stand: 07.04.2016
\bibitem{pypi}PyPi (2016), \emph{PyPi - the Python Package Index :
    Python Package Index}, Online im Internet:
  \url{https://pypi.python.org/pypi}, Stand: 07.04.2016
\bibitem{chuck}ChucK (2016), \emph{ChucK => Strongly-timed, On-the-fly
    Music Programming Language}, Online im Internet:
  \url{http://chuck.cs.princeton.edu/}, Stand: 07.04.2016
\bibitem{osc}Open Sound Control (2016), \emph{opensoundcontrol.org |
    an Enabling Encoding for Media Applications}, Online im Internet:
  \url{http://opensoundcontrol.org/}, Stand: 07.04.2016
\bibitem{leapmotion-community}Leap Motion Community (2013), \emph{Leap
    Motion support on Raspbmc (Raspberry Pi) - Customer Support - Leap
    Motion Community}, Online im Internet:
  \url{https://community.leapmotion.com/t/leap-motion-support-on-raspbmc-raspberry-pi/155},
  Stand: 07.04.2016
\bibitem{leapmotion-blog}Leap Motion Blog (2015), \emph{How to
    Integrate Leap Motion with Arduino \& Raspberry Pi}, Online im
  Internet:
  \url{http://blog.leapmotion.com/integrate-leap-motion-arduino-raspberry-pi/},
  Stand: 07.04.2016
\bibitem{pypi-gpio}PyPi RPi.GPIO (2016), \emph{RPi.GPIO 0.6.2 : Python
    Package Index}, Online im Internet:
  \url{https://pypi.python.org/pypi/RPi.GPIO}, Stand: 07.04.2016
\bibitem{leapmotion-developer}Leap Motion Developer (2016),
  \emph{Python SDK Documentation - Leap Motion Python SDK v2.3
    documentation}, Online im Internet:
  \url{https://developer.leapmotion.com/documentation/python/index.html},
  Stand: 07.04.2016
\bibitem{flask}Armin Ronacher (2016), \emph{Welcome | Flask (A Python
    Microframework)}, Online im Internet:
  \url{http://flask.pocoo.org/}, Stand: 07.04.2016
\bibitem{requests}Kenneth Reitz (2016), \emph{Requests: HTTP for
    Humans - Requests 2.9.1 documentation}, Online im Internet:
  \url{http://docs.python-requests.org/en/master/}, Stand: 07.04.2016
\bibitem{supercollider}SuperCollider (2016), \emph{SuperCollider »
    SuperCollider}, Online im Internet:
  \url{https://supercollider.github.io/}, Stand: 07.04.2016
\bibitem{csound}Csound (2016), \emph{Home | Csound Community}, Online
  im Internet: \url{https://csound.github.io/}, Stand: 07.04.2016
\bibitem{pyosc}pyOSC (2010), \emph{pyOSC - V2\_Lab Projects}, Online im
  Internet: \url{https://trac.v2.nl/wiki/pyOSC}, Stand: 07.04.2016

\end{thebibliography}

\end{document}
